\documentclass[onesided]{ccg-pset}

\course{MATH 6310}
\psnum{(REVISION TO PROBLEM 6 MIDTERM 1)}
\author{Colton Grainger}
\date{\today}

\begin{document}
\maketitle

\begin{enumerate}
\item[6] (``Escape to infinite frequency'') Consider the real line as a locally compact topological space with the Lebesgue measure: $(\R, \sM, \mu)$. If some function $f \colon \R \to \R$ is an absolutely integrable, then 
\begin{equation}
    \label{escape-to-infinite-frequency}
    \lim_{n\to\infty} \int_\R f(x)\cos(nx)\dd{x} = 0.
\end{equation}

\begin{proof}
We proved in lecture (more or less) that compactly supported piecewise constant functions are dense in 
\[L^1(\R):=\set{\text{absolutely integrable functions $\R \to \R$}}/\sim \qq{where} f \sim g \text{ iff } \int_\R(f-g) = 0.\] 
This density provides an approximation of $f$ with a sequence of compactly supported piecewise constant functions
\[
\set{h_m \colon \R \to \R}_{m  \in \N}
\]
such that 

\begin{enumerate}
\item the support of each $h_m$ contained in the closed interval $F_m :=[-2\pi m,2\pi m]$,
\item $\abs{h_m} \le \abs{f}$ for all $m \in \N$, and
\item the $h_m$ converge to $f$ in $L^1$, i.e.,
\begin{equation}
    \label{norm-zero}
    \int_\R (f - h_m) \dd{x} =: \norm{f - h_m}_1 \to 0 \qq{as} m \to \infty
\end{equation}
\end{enumerate}

Note that the $F_m$ have finite Lebesgue measure and have union $\cup_m F_m = \R$. Moreover, because $f \in L^1(\R)$ is finite a.e., the approximations $h_m$ are finite a.e. as well. 

To prove \eqref{escape-to-infinite-frequency}, by application of the approximation in \eqref{norm-zero} it suffices to fix $m \in \N$ and show that the product \[g_{n,m} \colon \R \to \R \qq{defined by} g_{m,n}(x) = h_m(x)\cos(nx)\] satisfies
\[
    \lim_{n\to\infty} \int_{F_m} g_{m,n} = 0.
\]
So fix $m \in \N$. Suppose for contradiction that $\lim_{n\to\infty} \int_\R g_{m,n} > 0$. Then we can decompose $h_m$ into countably many orthogonal components such that $\abs{h(x)} \ge \sum \abs{c_n \cos(nx)}$, where $c_n = \int_{F_m} g_{m,n}$. Because each $\cos(nx)$ is supported in $F_m$ on a set of fixed positive measure (namely $\frac{1}{2}m(F_m)$), if $c_n \not\to 0$ then $h_m \not\in L^1$, which is absurd, as $\abs{f} \ge \abs{h_m}$. The proposition follows.
\end{proof}

\end{enumerate}

\end{document}
