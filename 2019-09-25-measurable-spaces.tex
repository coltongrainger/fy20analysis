\documentclass[onesided]{ccg-pset}

\course{MATH 6310}
\psnum{3}
\author{Colton Grainger}
\date{\today}

\newcommand{\sB}{\mathscr{B}} 
\begin{document}
\maketitle

\begin{enumerate}
\setcounter{enumi}{16}

\item \emph{Note.} I have interpreted problem 2.17 from \cite{RF10} as an allusion to the \term{inner regularity} property
\begin{equation*}
    m(E) =\sup \{ \mu(K): K \subset E, \hbox{ compact} \}
\end{equation*}
and the \term{outer regularity} property
\begin{equation*}
    m(E)=\inf\{\mu(U): E \subset U,\hbox{ open}\} 
\end{equation*}
for the Lebesgue measure. ``Combined with the fact that $m$ is locally finite, this implies that the Lebesgue measure $m$ is a Radon measure.''%
    \footnote{%
    \url{https://terrytao.wordpress.com/2009/01/01/245b-notes-0-a-quick-review-of-measure-and-integration-theory}
    }

\begin{prop*}[]
    \label{prop:17}
    Let $m \colon \sM \to [0,\infty]$ be the complete Lebesgue measure on $\R$ associated to the identity function $\R \xrightarrow{\id} \R$. A set $E \subset \R$ is Lebesgue measurable iff for each $\epsilon > 0$ there's a compact set $K$ and an open set $U$ such that 
    \begin{equation}
        \label{closed-open-approx}
        K \subset E \subset U \qq{and} m(U \setminus K) < \epsilon.
    \end{equation}
\end{prop*}

\begin{proof}

\end{proof}

\item \emph{Note.}
    I have interpreted problem 2.18 from \cite{RF10} as a suggestion to verify Littlewood's first principle, by taking for granted
    (i) Carathéodory's extension theorem,
    (ii) the fact that \cite{Fol99} if $F \colon \R \to \R$ is any increasing, right continuous function, there is a unique Borel measure $\mu_F$ on $\R$ such that $\mu_F \paren{\{x \in \R:a< x\le b\}} = F(b) - F(a)$ for all $a,b \in \R$, and
    (iii) the knowledge that the completion of the Borel measure $\mu_F$ is the Lebesgue-Stieltjes measure associated to $F$, whose domain is the collection $\sM_F$ (or $\mathfrak{M}$ in the notation of \cite{RF10}) of Lebesgue measurable sets.

\begin{prop*}
    \label{prop:18}
    Suppose $\mu \colon \sM \to [0,\infty]$ is a complete Lebesgue-Stieltjes measure on $\R$ associated to an increasing, right continuous function $F$. If $E \subset \R$ has \emph{finite} measure with respect to $\mu$, then $E$ can be approximated by 
    \begin{enumerate}
        \item $E = V\setminus N_1$ where $V$ is a $G_\delta$ set and $\mu(N_1) = 0$, and
        \item $E = H \cup N_2$ where $H$ is an $F_\sigma$ set and $\mu(N_2) = 0$.
    \end{enumerate} 
\end{prop*}

\begin{proof}
    
\end{proof}

\setcounter{enumi}{23}
\item \emph{To prove.} If $A$ and $B$ are Lebesgue measurable sets, then $m(A \cup B) + m(A \cap B) = m(A) + m(B)$.
\begin{proof}

\end{proof}

\setcounter{enumi}{38}
\item  \emph{Given.} Let $F$ be the subset of $\bkt{0,1}$ constructed in the same manner as the Cantor set%
    \footnote{%
    Hence $F$ is to be called a \term{generalized Cantor set}.
    }
except that each of the intervals removed at the $n$th deletion stage has length $\alpha3^{-n}$ with $0 < \alpha < 1$. 
\begin{prop*}[]
    \label{prop:39}\hfill
    \begin{enumerate}
        \item $F$ is closed in $\R$
        \item $\bkt{0,1} \setminus F$ is dense in $[0,1]$
        \item the Lebesgue measure of $F$ is computed $m(F) = 1 - \alpha$.
    \end{enumerate}
\end{prop*}
\begin{proof}

\end{proof}

\item \emph{To prove.} There is an open set of real numbers that has boundary of positive measure, namely $\Cl([0,1] \setminus F) \cap F$.
\begin{proof}

\end{proof}
\end{enumerate}

\bibliography{/home/colton/coltongrainger.bib}
\bibliographystyle{alpha}
\end{document}
